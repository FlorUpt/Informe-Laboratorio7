\documentclass[12pt,letterpaper]{article}
\usepackage[latin1]{inputenc}
\usepackage[spanish]{babel}
\usepackage{graphicx}
\usepackage[left=2cm,right=2cm,top=2cm,bottom=2cm]{geometry}
\usepackage{graphicx} % figuras
\usepackage{subfigure} % subfiguras
\usepackage{float} % para usar [H]
\usepackage{amsmath}
\usepackage{txfonts}
\usepackage{stackrel} 
\usepackage[latin1]{inputenc}
\usepackage{multirow}
\usepackage{enumerate} % enumerados
\renewcommand{\labelitemi}{$-$}
\renewcommand{\labelitemii}{$\cdot$}
\author{Nelia Escalante}
\title{Caratula}
\begin{document}

\title{Caratula}

\begin{titlepage}
\begin{center}
\large{UNIVERSIDAD PRIVADA DE TACNA}\\
\vspace*{-0.025in}
\begin{figure}[htb]
\begin{center}
\includegraphics[width=11cm]{./IMG/logo}
\end{center}
\end{figure}
\Large INGENIERIA DE SISTEMAS  \\

\vspace*{0.5in}
\begin{large}
TITULO:\\
\end{large}

\vspace*{0.1in}
\begin{Large}
\textbf{Creacion de objetos de Base de Datos Oracle} \\
\end{Large}

\vspace*{0.3in}
\begin{Large}
\textbf{CURSO:} \\
\end{Large}

\vspace*{0.1in}
\begin{large}
BASE DE DATOS II\\
\end{large}

\vspace*{0.3in}
\begin{Large}
\textbf{DOCENTE:} \\
\end{Large}

\vspace*{0.1in}
\begin{large}
 Ing. Patrick Cuadros Quiroga\\
\end{large}

\vspace*{0.2in}
\vspace*{0.1in}
\begin{large}
Integrantes: \\
\vspace{\baselineskip}
\begin{flushleft}
Salamanca Contreras, Fiorella Rosmery		\hfill	(2015053237) \\
Escalante Maron, Nelia 		\hfill	(2014049551) \\
Condori Gutierrez, Flor de Maria            	\hfill	(2015053227) \\
Coaquira Calizaya, Yerson      	\hfill	(2015053225) \\
Espinoza Caso, Lizbeth  		\hfill	(2011040667) \\
\end{flushleft}
\end{large}
\end{center}

\end{titlepage}

\newpage

	\begin{center}
		\Large CREACION DE OBJETOS DE BASE DE DATOS ORACLE
	\end{center}
	\vspace{\baselineskip}
	\vspace{\baselineskip}
	\textbf{\Large 1. Desarrollar el script de creacion de TableSpace(s) y archivos para la base de datos de su proyecto.}
	\\\\
	CREATE TABLESPACE USUARIO\_DATA DATAFILE '/upt/DatosBase/DT\_USUARIO.DBF' SIZE 200M;\\
	CREATE TEMPORARY TABLESPACE USUARIO\_TEMP TEMPFILE\\ '/upt/DatosBase/DT\_USUARIO\_TMP.DBF' SIZE 200M;\\
	CREATE TABLESPACE USUARIO\_INDEX DATAFILE '/upt/DatosBase/DT\_USUARIO\_IDX.DBF' SIZE 200M;\\\\ 	
	SELECT * FROM DBA\_DATA\_FILES;\\\\	
	CREATE ROLE ADMINISTRADOR\_BD;\\\\
	
	--OTORGARLE PERMISOS: \\ 
\\  
	GRANT CREATE SESSION TO ADMINISTRADOR\_BD;\\ 
	GRANT CREATE ANY TABLE TO ADMINISTRADOR\_BD;\\ 
	GRANT CREATE ROLE TO ADMINISTRADOR\_BD;\\ 
	GRANT CREATE USER TO ADMINISTRADOR\_BD;\\ 
	GRANT CREATE VIEW TO ADMINISTRADOR\_BD;\\ 
	GRANT CREATE ANY INDEX TO ADMINISTRADOR\_BD;\\ 
	GRANT CREATE TRIGGER TO ADMINISTRADOR\_BD;\\ 
	GRANT CREATE PROCEDURE TO ADMINISTRADOR\_BD;\\
	GRANT CREATE SEQUENCE TO ADMINISTRADOR\_BD;\\\\ 	

           --CREACION DE USUARIO:\\\\ 
	CREATE USER USUARIO\_BD IDENTIFIED BY USUARIO\_BD DEFAULT TABLESPACE USUARIO\_DATA TEMPORARY            TABLESPACE USUARIO\_TEMP;\\\\

	--DARLE PERMISOS AL  ADMINISTRADOR\_BD SOBRE TABLESPACES\\\\	
	GRANT UNLIMITED TABLESPACE TO USUARIO\_BD;\\\\

	\newpage
	\textbf{\Large 2. Desarrollar el script de creacion de objetos (entidades atributos llaves objetos).}
	\\\\

	- - CREACION DE LA TABLA CU.ESTUDIANTE\\\\
	\small
	CREATE TABLE CU.ESTUDIANTE(\\
	ID\_ESTUDIANTE NUMBER(3) CONSTRAINT PK\_ESTUDIANTE PRIMARY KEY,\\
	NOMBRE\_ESTUDIANTE VARCHAR2(100) CONSTRAINT NN\_ESTUDIANTE\_NOMBRE\_ESTUDIANTE NOT NULL,\\
	DIRECCION\_ESTUDIANTE VARCHAR2(100) CONSTRAINT NN\_ESTUDIANTE\_DIRECCION\_ESTUDIANTE NOT NULL,\\
	TELEFONO\_ESTUDIANTE NUMBER(9) CONSTRAINT NN\_ESTUDIANTE\_TELEFONO\_ESTUDIANTE NOT NULL,\\
	CORREO\_ESTUDIANTE VARCHAR(50) CONSTRAINT NN\_ESTUDIANTE\_CORREO\_ESTUDIANTE NOT NULL\\
	) TABLESPACE "PCUADROS";
\vspace{\baselineskip}
	
	\normalsize
	- - CREACION DE LA TABLA CU.PRESTAMO\\\\
	\small
	CREATE TABLE CU.PRESTAMO (
	ID\_PRESTAMO NUMBER(3) CONSTRAINT PK\_PRESTAMO PRIMARY KEY,\\
	FECHASALIDA\_PRESTAMO VARCHAR2(15) CONSTRAINT NN\_PRESTAMO\_FECHASALIDA\_PRESTAMO NOT NULL,\\
	FECHAENTRADA\_PRESTAMO VARCHAR2(15) CONSTRAINT NN\_PRESTAMO\_FECHAENTRADA\newline \_PRESTAMO NOT NULL,\\
	CONSTRAINT FK\_PRESTAMO\_X\_BIBLIOTECA FOREIGN KEY (ID\_BIBLIOTECA) REFERENCES \\ CU.BIBLIOTECA(ID\_BIBLIOTECA),\\
	CONSTRAINT FK\_PRESTAMO\_X\_EMPLEADO FOREIGN KEY (ID\_EMPLEADO) REFERENCES  \\ CU.EMPLEADO(ID\_EMPLEADO),\\
	CONSTRAINT FK\_PRESTAMO\_X\_RESERVA FOREIGN KEY (ID\_RESERVA) REFERENCES CU.RESER\newline VA(ID\_RESERVA)\\
	) TABLESPACE "PCUADROS";

	\vspace{\baselineskip}
\normalsize
	- - CREACION DE LA TABLA CU.RESERVA\\\\
	\small
	CREATE TABLE CU.RESERVA(
	ID\_RESERVA NUMBRE(3) CONSTRAINT PK\_RESERVA PRIMARY KEY,\\
	FECHARECOJO\_RESERVA CONSTRAINT NN\_RESERVA\_FECHARECOJO\_RESERVA NOT NULL,\\
	CONSTRAINT FK\_RESERVA\_X\_TIPOEJEMPLAR FOREIGN KEY (ID\_TIPOEJEMPLAR) REFERENCES \\ CU.BIBLIOTECA(ID\_TIPOEJEMPLAR),\\
	CONSTRAINT FK\_RESERVA\_X\_ESTUDIANTE FOREIGN KEY (ID\_ESTUDIANTE) REFERENCES \\   CU.ESTUDIANTE(ID\_ESTUDIANTE)\\
	) TABLESPACE "PCUADROS";\\\\

	\vspace{\baselineskip}

	\newpage
	\normalsize
	- - CREACION DE LA TABLA CU.BIBLIOTECA\\\\
	\small
	CREATE TABLE CU.BIBLIOTECA(
	ID\_BIBLIOTECA NUMBER(3) CONSTRAINT PK\_BIBLIOTECA PRIMARY KEY,\\
	NOM\_BIBLIOTECA VARCHAR2(100) CONSTRAINT NN\_BIBLIOTECA\_NOM\_BIBLIOTECA NOT NULL,\\
	DIRECC\_BIBLIOTECA VARCHAR2(100) CONSTRAINT NN\_BIBLIOTECA\_DIRECC\_BIBLIOTECA NOT NULL,\\
	TELEFO\_BIBLIOTECA NUMBER(9) CONSTRAINT NN\_BIBLIOTECA\_TELEFO\_BIBLIOTECA NOT NULL,\\
	WEB\_BIBLIOTECA VARCHAR2(100) CONSTRAINT NN\_BIBLIOTECA\_WEB\_BIBLIOTECA NOT NULL\\
	) TABLESPACE "PCUADROS";
\end{document}